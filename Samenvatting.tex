\begin{otherlanguage}{dutch}
Publieke mentale gezondheid kan worden gedefineerd als ``een benadering op basis van de publieke gezondheidszorg van de geestelijke gezondheid en mentale gezondheid van de populatie \parencite{Davies_2014}.'' Publieke mentale gezondheid bestaat uit drie overlappende domeinen: Aandacht voor mentale gezondheid, preventie van psychische aandoeningen, en behandeling en revalidatie. Starpunt van alle drie deze domeinen omvat meestal het meten en beoordelen van mentale gezondheid. Meten van mentale gezondheid is echter niet eenvoudig. De uitdagingen die hiermee gepaard gaan liggen soms voor de hand maar zijn soms ook meer verdekt. Deze uitdagingen zijn echter altijd belangrijk, aangezien conclusies binnen de publieke mentale gezondheid, zowel op individueel als collectief niveau, vaak gebaseerd zijn op metingen. Hierdoor kunnen de uitdagingen klein van aard lijken terwijl ze in de praktijk verstrekkende gevolgen hebben. Dit maakt deze uitdagingen tot potenti\"ele verborgen gebreken. Als (onderzoek naar) publieke mentale gezondheid (mede) gebaseerd wordt op meetinstrumenten met een onbekende blinde vlek, kan dit een negatieve impact hebben. Het hoofddoel van dit proefschrift was daarom:
\begin{itemize}
	\item Het identificeren van potenti\"ele verborgen gebreken en de impact hiervan evalueren op de publieke mentale gezondheid
\end{itemize} 
In deze samenvatting worden de motivering, methodes, en resultaten zoals gebruikt in de verschillende hoofdstukken kort uiteengezet en besproken.

In hoofdstuk \ref{ch:intro} werd het concept publieke mentale gezondheid en het meten ervan voor het eerst ge\"introduceerd. De introductie richtte zich op vier belangrijke punten: (1) Mentale gezondheid is een continu\"um in plaats van een dichotome representatie (ziek/gezond). Zelfs als een dichotome representatie wordt gebruikt, of een andere vorm van categorisatie, is het onderliggende construct not steeds continu. (2) Dit onderliggende continu\"um is latent van aard. Dit wil zeggen dat publieke mentale gezondheid geen direct meetbare (manifeste) fenomenen kent. Daarom moet publieke mentale gezondheid worden gemeten aan de hand van indicatoren. Meetinstrumenten gebruiken dergelijke indicatoren om het onderliggende construct te omvatten. Het bepalen van de kwaliteit van dergelijke instrumenten is niet eenvoudig. De kwaliteit van meetinstrumenten wordt daarnaast voornamelijk getest door middel van verkennende en ad-hoc analyses. Deze aanpak heeft verschillende beperkingen wat er voor zorgt dat de meetinstrumenten kwetsbaar zijn voor verborgen gebreken. (3) Verborgen gebreken zijn ``actoren die resulteren in de schending van (belangrijke) kwaliteitscriteria van een meetinstrument binnen publieke mentale gezondheid, zonder dat dit (direct) duidelijk is.'' De verschillende aspecten in deze definitie werden verder bediscussieerd in de introductie van dit proefschrift. (4) \textbeta-Psychometrie kan worden toegepast om de kwaliteit van instrumenten die publieke mentale gezondheid meten te evalueren door middel van een a-priori theoretisch kader. Dit theoretisch kader maakt het mogelijk om confirmatief te testen. Hierdoor is het mogelijk om specifieke hypotheses te testen op de aan- of afwezigheid van verborgen gebreken.

Hoofdstuk \ref{ch:sdq} was de eerste empirische studie van het proefschrift. In dit hoofdstuk werd onderzocht of het afnemen van een meetinstrument in verschillende settings, de mate waarin sociale wenselijkheid invloed heeft op een meetinstrument veranderd. Dit zou er voor kunnen zorgen dat de vergelijking tussen deze verschillende settings niet valide is. Het `verborgen gebrek' dat hier werd onderzocht was dus het potenti\"ele verkeerde gebruik van scores (van een meetinstrument) tussen settings. In dit hoofdstuk werd gebruik gemaakt van de de Nederlandse versie van de `Strengths and Difficulties Questionnaire' \parencite[SDQ;][]{Goodman_1997}. Dit is een screeningsinstrument voor psychosociale problemen bij kinderen en adolescenten. Hierbij werd de afname in een individuele setting vergeleken met de afname in een collective setting.

De individuele setting kan worden gezien als de \'e\'en-op-\'e\'en toepassing van een instrument (bijv. jeugdgezondheidszorg). In deze setting is een instrument vooral geschikt voor screening \parencite{Vogels_2009, Widenfelt_2003, Goodman_2001}. Voor deze setting werd er in hoofdstuk \ref{ch:sdq} gebruikt gemaakt van data afkomstig van de jeugdgezondheidszorg. 6.594 vragenlijsten, afgenomen tijdens het periodiek gezondheidsonderzoek tijdens het tweede jaar van de middelbare school, zijn geanalyseerd. In de collectieve setting wordt de informatie van een meetinstrument juist geaggregeerd binnen een populatie \parencite{Roy_2006,Becker_2006,Ravens_2008}. Voor deze setting werd er gebruik gemaakt van informatie uit de Jeugdmonitor \parencite{Youth_2014a}. Deze monitor bevatte onder andere de SDQ als een indicator van psychosociaal welbevinden. Dit resulteerde in de analyse van 4.613 vragenlijsten. Aangezien beide steekproeven uit dezelfde populatie voortkwamen werd de hypothese geformuleerd dat eventuele significante verschillen een gevolg zijn van verschillen in de setting van de afname.

Confirmatieve factor analyses (CFA) lieten zien dat er geen significante verschillen waren in het onderliggende construct (\textit{psychosociale problemen}) tussen de twee steekproeven. Kinderen in de individuele setting, scoorden desalniettemin lager wat betreft totaal probleemgedrag en de onderliggende subschalen vergeleken met de kinderen in de collectieve setting. Om deze resultaten verder te kwantificeren werd er gekeken naar de vergelijkbaarheid van de scores in de verschillende settings. Dit werd onderzocht door het toepassen van afkappunten, gebaseerd op de collectieve setting, binnen de individuele setting. Deze procedure was vergelijkbaar met staand beleid aangaande het gebruik van afkappunten van de SDQ binnen de jeugdgezondheidszorg. Hierbij resulteerde het gebruik van het 90\textsuperscript{e} percentiel uit de collectieve setting binnen de individuele setting in slechts een klein aantal gevallen met een afwijkende score (2 tot 3\%). Dit terwijl 10\% zou worden verwacht -- als de setting geen impact zou hebben op de afname. 

Deze resultaten lieten zien dat de SDQ dezelfde connotatie heeft in de individuele en collectieve setting. Het verschil in de gemiddelde scores laten echter zien dat er een verschil zit in de bereidwilligheid om vakje aan te kruisen bij afname in de verschillende settings. Dit geobserveerde structurele verschil ondermijnt de validiteit van onderling gebruik van scores tussen settings. De toepassing van de afkappunten, afkomstig uit de collectieve setting, binnen de individuele setting zou dan ook resulteren in conclusies die niet valide zijn. Om deze afkappunten correct toe te passen zullen ze daarom ook afzonderlijk binnen iedere setting moeten worden vastgesteld.

De volgende drie hoofdstukken (\ref{ch:mei}-\ref{ch:tso}) gebruikten allemaal dezelfde populatie, de Maastrichtse Cohort Studie (MCS). Binnen de verschillende hoofdstukken werd er echter wel gebruik gemaakt van verschillende selecties binnen deze populatie. De MCS is opgezet in mei 1998 en betrof toen 12.140 deelnemers van 45 verschillende bedrijven. Gedurende deze baseline meting (T0) waren alle deelnemers tussen 18 en 65 jaar oud, waarvan 8.840 man (73\%) en 3.255 vrouw \parencite{Kant_2003, Mohren_2007}. Voor meer details aangaande de MCS zie \textcite{Mohren_2007}. 

Naast dezelfde steekproef deelden deze hoofdstukken ook de uitkomstmaten die bestudeerd werden: herstelbehoefte en (langdurige) vermoeidheid. Herstelbehoefte is een uitkomstmaat dat zich vooral richt op de behoefte en duur van herstel naar een dag werken. De normale herstel cyclus is idealiter afgerond na het werk maar tenminste voor het begin van de volgende werkdag \parencite{Veldhoven_2008}. Langdurige vermoeidheid is de aanwezigheid van vermoeidheid, inclusief een overweldigend gevoel van vermoeidheid, voor een langere periode \parencite{Bultmann_2000,Kalkman_2008}. Beide aspecten zijn belangrijke uitkomstmaten binnen de publieke mentale gezondheid. Daarnaast kunnen ze een belangrijke rol spelen bij preventieve strategi\"en die gericht zijn op het voorkomen of beperken van negatieve (gezondheids-) uitkomsten \parencite[bijv.,][]{Croon_2003,Amelsvoort_2003,Raeve_2009,Amelsvoort_2002,Silva_2012}. Herstelbehoefte werd gemeten met de herstelbehoefteschaal \parencite{Veldhoven_2003} en langdurige vermoeidheid met de Checklist Individuele Spankracht \parencite{Vercoulen_1994,Beurskens_2000}. De herstelbehoefteschaal en de CIS zijn beide gevalideerde meetinstrumenten met goede psychometrische eigenschappen \parencite[bijv.,][]{Veldhoven_2003,Bultmann_2000,Beurskens_2000,Sluiter_2001,Amelsvoort_2003}. Ook al laten deze studies zien dat het valide meetinstrumenten betreft, verschillende onderliggende assumpties (verborgen gebreken) zijn niet adequaat onderzocht. 

%Chapter MEI
In het eerste hoofdstuk van deze `trilogie', hoofdstuk \ref{ch:mei}, werd onderzocht of de herstelbehoefteschaal en de CIS al dan niet variabel functioneren in verschillende subgroepen. Deze assumptie is een typisch verborgen gebrek aangezien schending van de assumptie resulteert in een niet valide vergelijking tussen groepen. Afwezigheid van meetinvariantie tussen groepen is echter moeilijk vast te stellen met klassieke psychometrische analyses \parencite{Vandenberg_2000,Millsap_2011}. Met CFA werd daarom onderzocht of de factor structuur vergelijkbaar was tussen verschillende subgroepen. Deze subgroepen waren gebaseerd op demografische factoren (bijv. geslacht), persoonlijke factor (bijv. ervaren gezondheid), en werk gerelateerde factoren (bijv. soort dienst). Om dit te analyseren werd eerst de onderliggende factor structuur ge\"evalueerd. Ook al hadden vorige studies deze factor structuur al op een post-hoc basis bekeken, dit was de eerste studie die CFA gebruikte om de factor structuur te evalueren op basis van een a-priori model. Met een aantal kleine aanpassingen liet de gehypothetiseerde 1-factor structuur van de herstelbehoefteschaal een goed passend model zien \parencite{Veldhoven_2008}. Ook voor de CIS liet de gehypothetiseerde 4-factor structuur met een aantal kleine aanpassingen een goed passend model zien \parencite{Janssen_2003}. Alle modellen werden vervolgens vergeleken met deze baseline-modellen. Bij deze vergelijking was er geen aanwijzing voor meetvariantie tussen groepen. Hoofdstuk \ref{ch:mei}, toont daarmee aan dat op een a-priori basis de gehypothetiseerde factor structuur kan worden bevestigd. Daarnaast toont dit hoofdstuk aan dat deze factor structuur vergelijkbaar is voor een groot aantal subgroepen die vaak worden vergeleken binnen de publieke mentale gezondheid.

%Chapter Stab
Het hierop volgende hoofdstuk, hoofdstuk \ref{ch:stab}, ging dieper in op deze resultaten door te onderzoeken of er ook sprake was van meerinvariantie over tijd. Dit is een belangrijke assumptie bij meetinstrumenten die vaak niet getest wordt -- hierdoor een perfect illustratie van een `verborgen gebrek'. Dergelijke stabiliteit van metingen over tijd is cruciaal voor het adequaat monitoren van bijvoorbeeld herstelbehoefte en langdurige vermoeidheid. Als meetvariantie namelijk aanwezig is dan kunnen scores niet inhoudelijk met elkaar worden vergeleken waardoor kwalitatieve monitoring onmogelijk is. Hoofdstuk \ref{ch:stab} liet echter zien dat zowel de CIS als de herstelbehoefteschaal geen meetvariantie hadden over tijd. Hierdoor is het dus mogelijk om deze belangrijke concepten the monitoren en evalueren. Daarnaast werd er ook gekeken naar het kleinste verschil in score dat kan worden onderscheiden van meetfouten. Dit kan dus worden gezien als een significant verschil binnen een persoon over tijd. Deze analyse liet zien dat vooral bij de herstelbehoefteschaal het moeilijk is om te differenti\"eren tussen werknemers met een verhoogde herstelbehoefte. De betrouwbaarheid van beide meetinstrumenten was erg goed. Dit kwam echter deels door het vloer effect dat optrad bij de herstelbehoefteschaal. De aanbeveling is daarom dat idealiter dergelijke meetinstrumenten worden ge\"incorporeerd binnen een bredere testbatterij om zodoende effectief te effecten van de psychosociale werkomgeving te meten. 

%Chapter TSO
Het laatste hoofdstuk van deze trilogie, hoofdstuk \ref{ch:tso}, ging dieper in op de longitudinale stabiliteit van de CIS. Om dit goed te kunnen duiden was het van belang om de geobserveerde toestand van vermoeidheid te splitsen in de component die varieert over tijd en een component die stabiel is over tijd. Daarnaast werd er ook gekeken naar de relatie met andere negatieve gezondheidsuitkomsten zoals ziekteverzuim. Door gebruik te makken van \textit{Trait-State-Occasion} (TSO) model  \parencite{Prenoveau_2011,Prenoveau_2016} in dit hoofdstuk werd aangetoond dat de tijd-invariante component van vermoeidheid 71\% van de geobserveerde vermoeidheid verklaarde. De tijd-variante component daarentegen verklaarde slechts 28\%. De overgebleven 1\% werd verklaard door het vorige meetmoment. Deze resultaten lieten zien dat vermoeidheid, of toch in ieder geval de CIS, een hoge mate van stabiliteit over tijd liet zien. Een onderliggende aanname bij vermoeidheid lijkt echter te zijn dat het redelijk reactief is op veranderingen in de omgeving van een persoon. Dit hoofdstuk toonde echter aan dat binnen een steekproef van werknemers deze assumptie geen standhoudt. Als deze assumptie niet op waarde wordt geschat kan het leiden tot verkeerde beoordeling van de resultaten en hierdoor wordt een valide conclusie ook belemmerd \parencite{Hamaker_2015}.

Naast deze decompositie om de geobserveerde toestand van vermoeidheid te bestuderen, gemeten met de CIS, werd in dit hoofdstuk ook de relatie tussen de verschillende componenten en ziekteverzuim geanalyseerd. Er werd daarom gekeken naar de tijd-invariante en tijd-variante component in relatie met ziekteverzuim. Uit de analyses bleek dat alleen de tijd-invariante component ziekteverzuim voorspelde binnen een 2-jaar durend tijdsinterval. Dit was zo voor drie verschillende operationalisaties van ziekteverzuim die werden bestudeerd: langdurige periode (opeenvolgende ziektedagen, $\geq$ 21 dagen), duur (totaal aantal ziektedagen, $\geq$ 42 dagen) en frequentie (aantal periodes; $\geq$ drie periodes). Er kon dus geconcludeerd worden dat vermoeidheid een aanzienlijke tijd-invariante component heeft. Aangezien het ook deze component was die negatieve gezondheidseffecten voorspelde, is het belangrijk dat preventieve maatregelen en behandeling deze decompositie in ogenschouw nemen.   

Hoofdstuk \ref{ch:brmsea} bevatte een simulatie studie die was opgezet om een nieuw ontworpen passings-index binnen Bayesiaanse CFA te valideren. Daar waar alle vorige hoofdstukken een frequentistisch kader gebruikten, kan Bayesiaanse CFA verschillende voordelen hebben binnen bepaalde situaties in vergelijking met frequentistische CFA modellen. Als er een bijvoorbeeld een veelvoud aan subgroepen (bijv. >40) is, kan het binnen een frequentistisch kader erg moeilijk zijn om de bron van eventuele verschillen te `ontdekken'. Bayesiaanse CFA heeft goede methoden om met dergelijke problemen om te gaan, ook binnen als de steekproef groot is \parencite{Schoot_2014}. Binnen dergelijke Bayesiaanse CFA-modellen, met een grote steekproef, ontbreekt het echter een bruikbare passings-index die een objectieve beoordeling biedt voor de kwaliteit van een model. Op basis van de root mean square error of approximation ($RMSEA$) uit het frequentistische kader werd daarom de Bayesiaanse $RMSEA$ ($BRMSEA$) voorgesteld. Dit hoofdstuk toonde aan dat deze $BRMSEA$ zeer goed presteert in grote steekproeven en succesvol modellen accepteert met geen of een kleine misspecificatie en modellen afkeurt met significante misspecificatie(s). De toegevoegde waarde van de $BRMSEA$ werd verder aangetoond in een empirische illustratie met behulp van de MCS. Deze illustratie onderzocht de factorstructuur van een subschaal van de Job Content Questionnaire \parencite{Karasek_1985}. Er werd aangetoond dat de klassieke passings-index niet-informatief werd binnen grote steekproeven, terwijl de $BRMSEA$ met succesvol detecteerde dat de voorgestelde factorstructuur van het Bayesiaanse CFA-model een goed passend model was. De $BRMSEA$ maakt het mogelijk om de aanwezigheid van meetinvariantie te onderzoeken bij meetinstrument tussen een groot aantal subgroepen (bijv. verschillende beroepen) door middel van specifieke methodes binnen de Bayesiaanse CFA \parencite{Schoot_2013a}. Concluderend kan worden vastgesteld dat de $BRMSEA$ goed geschikt is om in BSEM modellen met een grote steekproef de kwaliteit van een model te evalueren door rekening te houden met deze steekproefomvang en de complexiteit van het model.

In hoofdstuk \ref{ch:disc}, de algemene discussie, worden drie belangrijke aspecten benadrukt om goed om te gaan met `verborgen gebreken' in relatie tot het meten van de publieke mentale gezondheid. Het eerste aspect betrof de identificatie/opsoring van verborgen gebreken. Hierbij werd ge\"illustreerd dat een actieve manier van modelleren een absolute noodzaak is om met succes verborgen gebreken binnen de publieke mentale gezondheid te identificeren. Het tweede aspect betrof het vergroten van het bewustzijn voor dergelijke verborgen gebreken. Zonder dit bewustzijn zullen verborgen gebreken altijd de overhand hebben. De discussie ging in op specifieke methoden om dit bewustzijn te vergroten (bijv. onderwijs en kritische beoordeling van wetenschappelijk werk). Het derde en laatste aspect betrof de aanpak en preventie van potenti\"ele verborgen gebreken aangaande het meten van publieke mentale gezondheid. Naast deze aspecten werden enkele conceptuele eigenschappen van \textbeta-psychometrie behandeld. Hierbij werd zowel gekeken naar meer generieke eigenschappen als eigenschappen die specifiek voor dit proefschrift van belang waren. Hierbij werd aangetoond dat andere methodes een alternatief perspectief kunnen bieden op de conclusies en en bevindingen van dit proefschrift. Verder werd benadrukt dat een verscheidenheid aan instrumenten en populaties was bestudeerd. Ook zijn enkele beperkingen van de verschillende hoofdstukken aan de orde geweest. De ontwikkeling op het gebied van \textbeta-psychometrie met betrekking tot de implementatie van Bayesiaanse statistiek werd ook aangestipt. Momenteel zijn passings-indexen hierbinnen echter nog niet erg bruikbaar bij grote steekproeven -- welke typisch zijn voor de publieke mentale gezondheid. De nieuw ontworpen $BRMSEA$ in hoofdstuk \ref{ch:brmsea} kan hier verandering in brengen.

Dit proefschrift toont aan dat `verborgen gebreken' met succes kunnen worden ge\"identificeerd door middel van \textbeta-psychometrics. Met behulp van een vari\"eteit aan geavanceerde analyse methodes maakt dit proefschrift duidelijk dat bewustzijn voor verborgen gebreken cruciaal is, omdat metingen aan de basis staan van bijna alle aspecten binnen de publieke mentale gezondheid, betreffende onderzoek, klinische praktijk en beleidsvorming. Aangezien \textbeta-psychometrics gebaseerd is op a-priori modellen is het van groot belang dat de onderliggende aannames van een meetinstrument worden ge\"expliciteerd alvorens het mogelijk is de kwaliteit van een meetinstrument te beoordelen. Als een dergelijk begripsniveau met betrekking tot een meetinstrument aanwezig is, kunnen 'verborgen gebreken' worden ge\"identificeerd. Als verborgen gebreken echter de overhand hebben, zullen ze negatief effect sorteren binnen de gehele publieke mentale gezondheid, waardoor een geldig oordeel en besluitvorming wordt belemmerd.

\sloppy
\section*{References}
\printbibliography[heading=none]

\fussy
\end{otherlanguage}
%END CHAPTER S [Summary]